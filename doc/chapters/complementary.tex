% !TeX root = ../main.tex

\chapter{存储器层次结构回顾}

\textbf{缓存}:地址离开处理器后遇到的最高一级或第一级存储器层次结构,局部性原理支撑了缓存的高性能的理论逻辑。其中\textbf{时间局部性}指当前访问的数据在不久的将来极有可能还会再次用到,\textbf{空间局部性}指当前用到的数据附近的数据(同一个数据块)也极有可能会用到。

\textbf{虚拟存储器}:一些数据可以存储在磁盘上,地址空间被划分为固定大小的块(页),任何时候页要么在主存储器上,要么在磁盘上。当地址页不在缓存也不再主存储器上时发生页错误,要把整个页从磁盘加载到主存储器上。页错误消耗的时间过长,处理器一般会切换任务。
\section{缓存及缓存性能}

\chapter{存储器层次结构回顾}




